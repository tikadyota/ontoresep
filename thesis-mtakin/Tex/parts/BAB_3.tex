\chapter{LANDASAN TEORI}
% -------------------------------------------------------------------------------------------------------------------
% Konten:
% 1. Semantik web
% 2. Pemrosesan Bahasa Alami
% 3. Stemming ??? 
% 	a. algoritma stemming bahasa Indonesia ???
% 4. Ontologi
% 	a. Metode pembuatan ontologi
% 5. RDF
% 6. OWL Ontology
% 8. Ontologi Reasoning
% -------------------------------------------------------------------------------------------------------------------
Teknologi semantik web merupakan perluasan dari teknologi web yang telah ada saat ini atau lebih tepatnya adalah sebagai solusi dari kelemahan yang dimiliki oleh web saat ini. Seperti diketahui bahasa markup yang digunakan oleh web saat ini seperti HTML dan CSS hanya berfokus pada cara menampilkan dokumen saja tanpa memahami makna dari apa yang ditampilkannya. Akbiatnya, komputer tidak dapat memproses data lebih jauh lagi, seperti misalnya melakukan penyimpulan (inferensi) serta bertukar informasi antar website tanpa harus melibatkan peran pengguna.

\citet{liyang_yu} menyebutkan bahwa tujuan dari semantik web yaitu untuk mendapatkan informasi sebanyak mungkin mengenai sesuatu yang ingin kita gali, sesuatu di sini dapat berupa orang, acara, produk dan lain sebagainya hanya dengan melakukan query terhadap sekumupulan data yang sudah tersedia dalam website. Saat ini, kita tidak dapat mengetahui sebuah informasi yang terkandung dalam sebuah website hanya dengan melakukan query sederhana terhadap halaman web tersebut. Hal ini dikarenakan komputer tidak memahami infromasi yang ditampilkannya.

Penggunaan website saat ini sudah cukup beragam mulai dari membaca berita, berinteraksi dengan pengguna lain melalui media sosial maupun pencarian data dengan menggunakan mesin pencari seperti Google, Yahoo, Bing dan lain sebagainya. Melakukan pencarian data saat ini merupakan salah satu aktifitas yang paling banyak dilakukan oleh pengguna internet, namun dengan semakin banyaknya jumlah sumber data yang tersebar di internet mengakibatkan timbulnya permasalah baru yaitu susahnya mendapatkan informasi yang relevan dengan yang kita kehendaki, sebagai contoh misalnya kita ingin mencari infromasi mengenai program studi ilmu komputer Universitas Gadjah Mada dengan memasukkan kata kunci ``pascasarjana ilmu komputer UGM'' maka informasi yang akan kita dapatkan adalah tautan terhadap halaman-halaman yang mengandung kata-kata \emph{pascasarjana ilmu komputer ugm} yang jumlahnya dapat mencapai jutaan tautan. Hal ini tentunya akan menyebabkan proses untuk mendapatkan informasi yang sederhana seperti itu cukup memakan waktu karena selain kita harus membuka tautan yang diberikan mesin pencari tadi, kita juga harus membaca isi halaman tersebut untuk mengetahui apakah informasi yang disajikan sesuai dengan yang kita kehendaki atau tidak. Untuk itulah semantik web hadir memberikan jawaban atas persoalan ini.

\section{Pemrosesan Bahasa Alami}
Pemrosesan bahasa alami atau \emph{Natural Language Processing (NLP)} merupakan salah satu bidang yang tidak dapat dipisahkan dari sebuah sistem \emph{question answering}. Tujuan dari pemrosesan bahasa alami adalah untuk membuat model komputasi dari bahasa, sehingga memungkinkan manusia berinteraksi dengan komputer dengan menggunakan bahasa alami.

\citet*{kao_potet} menjelaskan bahwa pemrosesan bahasa alami (NLP) adalah usaha untuk mendapatkan representasi makna dari \emph{free text}. Pemrosesan bahasa alami merupakan bagian terdepan dalam sistem \emph{question answering}


\section{Gramatika}
Grammar suatu bahasa dapat dilihat sebagai suatu aturan yang menentukan apakah suatu kumpulan kata dapat diterima sebagai kalimat oleh bahasa tersebut. Sebuah bahasa L dapat dijelaskan sebagai se dari \emph{string}, dimana \emph{string} dibentuk dari bagian terkecil yang disebut dengan \emph{symbol}.

Sebuah \emph{grammar} G dinyatakan dalam persamaan 
\section{Parsing}
Parsing merupakan proses menganalisa sekumpulan kata berdasarkan grammar tertentu
\section{Ontologi}
Ontologi memiliki peranan penting dalam semantik web. Terdapat berbagai definisi ontologi dalam bidanng semantik web, menurut T.R Gruber melalui \citet*{antoniou}, ontologi adalah \emph{spesifikasi formal dari sebuh konseptualisasi}, sedangkan W3C melalui \citet{liyang_yu} mendefinisikan ontologi sebagai \emph{definisi formal dari sekumpulan term yang digunakan untuk mendeskripsikan dan merepeserentasikan sebuah domain tertentu.}

Ontologi berfungsi sebagai media untuk berbagi pengetahuan dan pemahaman terhadap sesuatu antra domain atau berbagi terminologi yang berbeda namun memliki makna yang sama, misalnya \emph{ZIP Code} sama dengan Kode Wilayah di Indonesia, dengan demikian apabila seseorang mencari dengan menggunakan kata kunci kode wilayah untuk suatu daerah di Amerika misalnya, maka komputer akan dapat memahai bahwa yang dimaksud adalah ZIP Code, demikian juga sebaliknya.

\subsection{Metode pengembangan ontologi}
\citet{fernandez_lopez} melalui \citet*{fonou_huisman} menyebutkan berbagai macam metode yang dapat digunakan untuk pengembangan ontologi, namun demikian \citet{noy_mcguinness} mengungkapkan bahwa tidak ada satu metode yang pasti dalam mengembangkan ontologi. Ia juga mengungkapkan sesungguhnya proses pembuatan ontologi adalah sebuah proses iteratif yang tidak dapat dikerjakan hanya dalam satu tahapan saja, bahkan sangat mungkin pengembangan ontologi terus berlanjut meskipun ontologi sudah digunakan. 

Pemilihan metode pengembangan tergantung pada masing-masing pengembang ontologi, seperti misalnya \citet*{fonou_huisman} memilih menggunakan metode yang dikembangkan oleh \citet*{uschold_king} dengan alasan bahwa metode ini lebih mudah dipahami bagi para pengembang ontologi pemula.

\citet{noy_mcguinness} menawarkan salah satu metode pengembangan ontologi yang didasakan pada pengalaman mereka dalam mengembangkan ontologi. Metode ini paling banyak digunakan dalam pengembangan ontologi. Secara umum tahapan yang harus dilalui dalam pengembangan ontologi adalah sebagai berikut:
\begin{itemize}
	\item Tentukan domain dan ruang lingkup \emph{scope} dari ontologi\\
	Untuk membantu dalam menentukan domain dan ruang lingkup dari ontologi yang akan dibangun, seorang ahli ontologi \emph{ontology engineer} harus dapat menjawab pertanyaan:
	\begin{itemize}
		\item Domain apa yang ingin di-cover oleh ontologi ini ?
		\item Akan digunakan untuk apa ontologi ini ?
		\item Pertanyaan seperti apa yang harus dapat dijawab oleh ontologi ini ?
	\end{itemize}
	Jawaban atas pertanyaan-pertanyaan tersebut mungkin saja dapat berubah selama proses pengembangan ontologi berlangsung, namun setidaknya dapat membantu untuk memastikan ontologi yang akan dibangun tidak keluar dari rung lingkup yang sudah ditetapkan.
	\item Gunakan ontologi yang sudah ada\\
	Sebelum mulai mengembangkan ontologi, ada baiknya untuk mencari apakah ontologi yang akan dibuat sudah pernah dibuat atau belum. Jika sudah ada, apabila memenuhi kriteria yang diinginkan maka sebaiknya menggunakan ontologi tersebut.
	\item Tentukan semua \emph{term} penting dalam ontologi\\
	Tentukan semua \emph{term} baik berupa kelas, objek properti maupun datatype property dari ontologi domain yang akan di-cover.
	\item Buat semua kelas dan strukturnya\\
	Pada tahapan ini, kelas-kelas yang akan di representasikan dalam domain dibuat terlebih dahulu, kemudian diikuti dengan membuat relasi antar kelas-kelas tersebut. Relasi disini termasuk struktur sub dan super kelas. Untuk menentukan struktur relasidapat menggunakan metode \emph{top-down, bottom-up} atau kombinasi keduanya.
	\item Buat properti kelas - \emph{slot}\\
	Setelah proses pembuatan kelas selesai, selanjutnya buat juga properti yang akan digunakan pada kelas-kelas yang sudah dibuat.
	\item Tentukan \emph{facet} dari \emph{slot}\\
	\emph{Facet} menjelaskan mengenai tipe nilai dari kelas, nilai yang diperbolehkan, jumlah yang diperbolehkan \emph{(cardinality)}.
	\item Buat anggota dari kelas \emph{(instance)}\\
	Langkah terakhir adalah membuat anggota atau \emph{instance} dari masing-masing kelas.
\end{itemize}
% \section{\emph{Resource Description Framework}}
\emph{Resource Description Framewok (RDF)} merupakan dasar pembentuk basis pengetahuan dalam semantik web, diajukan oleh W3C sebagai standar pada tahun 1999. RDF terdiri dari subjek, predikat dan objek yang kemudian dikenal dengan naman \emph{statement}. Subjek dapat berupa URI (Uniform Resource Identifier) yang berfungsi sebagai penanda atau \emph{identifier}, tidak menuntup kemungkinan sebuah subjek juga dapat berupa URL \emph{(Uniform Resource Locator)} yang merupakan bentuk khusus dari URI.  Predikat berupa URI yang menjelaskan hubungan antara subjek dengan objeknya sedangkan objek dapat berupa URI ataupun \emph{literal}. Sebuah pernyataan dalam RDF disebut dengan istilah \emph{triple}. secara grafis, bentuk \emph{statement} RDF dapat dilihat pada gambar \ref{fig:rdf_statement}.

\begin{figure}[!]
	\centering
	\includegraphics[trim = 29mm 141mm 40mm 0mm, clip, scale=0.6]{gambar}
	\caption{Struktur \emph{statement} RDF}
	\label{fig:rdf_statement}
\end{figure}

Sebagai contoh misalnya pernyataan ``Syamsul hobi badminton'', ``Syamsul memiliki saudara 2 orang'', pernyataan-pernyataan tersebut dapat dibentuk menjadi sebuah \emph{triple}. Pada pernyataan pertama, \emph{Syamsul} adalah merupakan subjek sedangkan \emph{hobi} merupakan sebuah predikat dan \emph{badminton} merupakan sebuah objek, sedangkan pada pernyataan kedua yang menjadi subjek adalah \emph{Syamsul} sedangkan predikat adalah \emph{memmiliki saudara} dan \emph{2 orang} merupakan objek. Jika objek pada pernyataan pertama di atas memerlukan penjelasan lebih lanjut, misalnya mengenai apa itu badminton, maka objek tersebut dapat berupa \emph{resource} dimana \emph{resource} tersebut membentuk triple-triple seperti terlihat pada gambar \ref{fig:rdf_multi_statement}. Pada pernyataan kedua, hanya dimungkinkan berupa literal karena tidak memerlukan penjelasan lebih lanjut mengenai objek itu sendiri.

Agar dapat di proses oleh komputer maka RDF triple harus dituliskan dalam bahasa atau sintak yang dapat dimengerti oleh komputer. Hingga saat ini bentuk penulisan RDF yang direkomendasikan oleh W3C adalah dalam bentuk XML dengan menggunakan namespace yang khusus RDF. Contoh statement di atas dapat kita serialisasi menjadi RDF sebagai berikut:
\lstinputlisting[firstline=1, lastline=7]{./parts/codeblock.xml}
\begin{figure}[!]
	\centering
	\includegraphics[trim = 0mm 0mm 0mm 93mm, clip, scale=0.55]{gambar}
	\caption{RDF dengan multi-statement}
	\label{fig:rdf_multi_statement}
\end{figure}
Selain XML/RDF, RDF juga dapat dituliskan dengan menggunakan Turtle sintaks sebagai berikut:
\lstinputlisting[firstline=9, lastline=11]{./parts/codeblock.xml}

Sintaks turtle lebih mudah dipahami oleh manusia, sehingga lebih mudah dibentuk. Meskipun sintak ini masih belum menjadi rekomendasi W3C namun besar kemungkinan kedepan juga akan menjadi rekomendasi karena sudah memiliki draft yang dapat dilihat di http://www.w3.org/TR/turtle/.
% \section{\emph{RDF Schema}}
Dalam kasus yang lebih kompleks, RDF tidak cukup kuat untuk menjelaskan semantik dari sebuah subjek yang sedang dijelaskan. Jika kita kembali pada contoh di atas, apabila kita ingin lebih jauh menjelaskan mengenai misalnya apa/siapa Syamsul Muttaqin, RDF tidak dapat menjelaskan hal ini \citep*{antoniou}. Untuk mengatasi hal ini, maka di diperkenalkan RDF Schema (RDFS).

\begin{figure}[h]
	\centering
	\includegraphics[trim = 0mm 0mm 0mm 32mm, clip, scale=0.55]{gambar}
	\caption{RDFS-statement}
	\label{fig:rdfs_statement}
\end{figure}

Sesuai dengan namanya, RDF Schema memberikan penjelasan lebih jauh mengenai objek yang sedang dibicarakan. Untuk itu RDFS diperkaya dengan beberapa penambahan namespace seperti rdfs:Class yang digunakan untuk menjelaskan tipe dari sebuah objek, rdfs:subClassOf yang merupakan turunan dari kelas, rdfs:domain, rdfs:range, serta beberapa penambahan lainnya. Gambar \ref{fig:rdfs_statement} menunjukkan RDFS statement, dimana \emph{foaf:Person} adalah kelas dan \emph{myRef:syamsul} merupakan \emph{instance} dari kelas \emph{Person}.
\section{OWL Ontologi}
OWL \emph{(Web Ontology Language)} dijadikan sebagai rekomendasi formal oleh W3C pada 10 februari 2004 \citep{liyang_yu}. OWL dirancang untuk kompatibel dengan sintak XML. OWL merupakan pengembangan RDF dan RDFS yang menjadi rekomendasi W3C sebelumnya, oleh karena itu secara sintaksis OWL kompatibel dengan sintak RDF dan RDF Schema.

Ide awal dari pengembangan OWL berdasarkan fakta bahwa RDF Schema belum cukup kuat dalam mereperesentasikan semantik dari sebuah \emph{statement} sehingga diperlukan definisi lebih lanjut. Definisi inilah yang kemudian diperkenalkan dalam OWL. \citet*{antoniou} menjelaskan beberapa model semantik yang tidak dapat dituangkan dalam RDF Schema diantaranya :
\begin{enumerate}
	\item Cakupan \emph{(scope)} dari sebuah properti. Sebagai contoh misalnya properti atau predikat \textit{memakan}, RDFS tidak dapat membatasi range cakupan properti ini hanya untuk kelas tertentu, misalnya kita tidak dapat menyebutkan ``Sapi hanya memakan rumput'', sementara sapi sendiri merupakan \emph{instance} dari kelas binatang, dimana kelas ini tidak hanya berisi sapi saja, namun juga dapat berisi kucing, sementara kucing tidak memakan rumput.
	\item \emph{Disjoint} antar kelas. RDFS hanya menjelaskan mengenai hirarki kelas--sub-kelas, ia tidak dapat membedakan apakah dua atau lebih kelas yang berbeda atau tidak. Sebagai contoh, misalnya kita ingin mendefinisikan kelas mobil dan motor adalah dua kelas yang berbeda, yang artinya apabila \emph{x} adalah instance dari kelas motor, maka \emph{x} tidak mungkin menjadi instance dari kelas mobil. RDFS tidak memiliki properti untuk menjelaskan hal ini, ia hanya dapat menjelaskan bahwa kedua kelas tersebut adalah merupakan sub-kelas dari kelas induk yaitu kendaraan.
	\item Kelas kombinasi. RDFS tidak dapat mendefinisikan sebuah kelas baru yang merupakan gabungan (union) dari dua atau lebih kelas lain. RDFS juga tidak dapat mendefinisikan bentuk kombinasi lain seperti isrisan atau \emph{intersection} ataupun complement dari dua buah kelas yang berbeda. Misalnya kelas Kendaraan adalah gabungan dari kelas Mobil dan Motor.
\end{enumerate}
Sebuah dokumen OWL terdiri dari elemen header, elemen kelas, elemen properti, elemen resktiksi properti, elemen properti khusus, serta elemen kombinasi boolean. 

\subsection{Elemen \emph{header}}
Sesuai dengan standar aturan XML dimana sebuah file terdiri dari sebuah elemen \emph{root}, elemen \emph{root} dari OWL adalah rdf:RDF dimana pada elemen \emph{root} ini dideklarasikan pula beberapa \emph{namspace} yang menjadi standar seperti terlihat pada contoh berikut:
\lstinputlisting[firstline=13, lastline=16]{./parts/codeblock.xml}

Header terdapat diantara elemen \texttt{<owl:Ontology> </owl:Ontology>}. Header berisi informasi mengenai OWL yang bersagkutan seperti informasi versi, keterangan dan lain sebagainya. Berikut ini adalah contoh bentuk dari header OWL
\lstinputlisting[firstline=18, lastline=22]{./parts/codeblock.xml}

\subsection{Elemen kelas}
Bagian selanjutnya adalah elemen kelas, bagian ini berada diantara elemen \texttt{<owl:Class></owl:Class>}. Sebuah kelas dapat terdiri dari beberapa sub kelas, seperti pada RDF Schema, apabila sebuah kelas merupakan sub dari kelas tertentu, maka definisinya dijelaskan di dalam elemen kelas tersebut. Sebagai contoh misalnya definisi kelas berikut:
\lstinputlisting[firstline=24, lastline=27]{./parts/codeblock.xml}

Elemen kelas di atas mendefinisikan sebuah kelas bernama laki-laki yang memiliki hubungan disjoint dengan kelas perempuan dan merupakan sub-kelas dari Person. OWL memiliki beberapa properti kelas selain disjoin seperti equivalentClass yang digunakan untuk menjelaskan ekuivalensi sebuah kelas dengan kelas tertentu, disjointUnion untuk menjelaskan sebuah kelas dijoint dengan beberapa buah kelas yang digabungkan dan lain sebagainya.

\subsection{Elemen properti}
Elemen properti adalah elemen yang menjelaskan mengenai predikat dari sebuah statement, dimana predikat ini menjelaskan hubungan antar kelas atau antar instance sebuah kelas dengan nilai dari properti instance tersebut. Oleh karena itu elemen properti terdiri dari dua jenis yaitu object property dan datatype property.

\emph{Datatype property} menjelaskan hubungan antara \textit{instance} sebuah kelas dengan properti dari \textit{instance} tersebut, misalnya properti \textbf{umur} menjelaskan hubungan antara \textbf{person1} dengan sebuah literal value \textbf{"28"}. 
\lstinputlisting[firstline=29, lastline=31]{./parts/codeblock.xml}

\emph{Object property} menjelaskan hubungan antara sebuah kelas dengan kelas lainnya, misalnya properti diampuOleh menjelaskan hubungan antara kelas dosen dengan kelas matakuliah. 
\lstinputlisting[firstline=33, lastline=37]{./parts/codeblock.xml}

OWL juga memungkinkan kita untuk mendefinisikan inverse dari sebuah properti. Dari contoh di atas, elemen \texttt{<owl:inverseOf rdf:resource="\#mengampu" />} menjelaskan bahwa properti \textbf{diampuOleh} memiliki properti inverse yaitu mengampu, dimana nilai \texttt{rdfs:domain} dan \texttt{rdfs:range} dari properti mengampu merupakan kebalikan dari nilai \texttt{rdfs:domain} dan \texttt{rdfs:range} yang dimiliki oleh properti \textbf{diampuOleh}.

\subsection{Sub-bahasa OWL}
Kemampuan OWL dalam membentuk ekspresi pengetahuan yang sangat lengkap memunculkan kendala dalam hal kemampuan komputer untuk melakukan \emph{reasoning}. Waktu komputasi yang dibutuhkan dalam proses reasoning dapat tidak terhingga, oleh karena itu kelompok kerja bidang ontologi di W3C seperti yang disebutkan oleh \citet*{mcguinness_vanharmelen} membagi OWL ontologi menjadi tiga buah sub bahasa berdasarkan batasan ekspresi logika yang dapat dibentuk yaitu:
\begin{enumerate}
	\item OWL-Full
	\item OWL-DL
	\item OWL-Lite
\end{enumerate}
OWL-Lite merupakan sub-bagian dari OWL-DL, demikian juga dengan OWL-DL merupakan sub-bagian dari OWL-Full. Gambar \ref{fig:owl_subset} menunjukkan ilustrasi dari \emph{subset} OWL.
\begin{figure}[h]
	\centering
	\includegraphics[width=0.5\textwidth]{owl_subset.jpg}
	\caption{Diagram venn \emph{subset} OWL 1}
	\label{fig:owl_subset}
\end{figure}
\section{OWL 2 Ontologi}
OWL terus dikembangkan seiring dengan semakin pesatnya perkembangan dan kebutuhan akan \emph{knowledge sharing}, oleh karena itu, kelompok kerja ontologi di W3C pada tahun 2012 menetapkan versi baru dari OWL yang disebut dengan OWL 2.0 atau OWL 2.

OWL 2 merupakan kelanjutan dari OWL 1.1 dengan beberapa penambahan dan perbaikan fitur. \ref{fig:owl_2_structure} memperlihatkan struktur dasar dari OWL 2.

Bagian atas pada gambar \ref{fig:owl_2_structure} menunjukkan format sintak yang dapat dipergunakan dalam menyusun ontologi dengan menggunakan bahasa OWL 2. Pada OWL 1, format yang dapat digunakan terbatas pada RDF/XML, sedangkan pada OWL 2 seperti yang terlihat dalam gambar \ref{fig:owl_2_structure} terdapat lima buah format yang dapat digunakan. Dari semua format tersebut, W3C hanya mewajibkan format RDF/XML sebagai format standar, sedangkan format lainnya berupa opsional saja.

\begin{figure}[h]
	\centering
	\includegraphics[width=1\textwidth]{owl_2_structure}
	\caption{Struktur OWL 2.0}
	\label{fig:owl_2_structure}
\end{figure}

Masing-masing format sintak memiliki kelebihan dan kekurangan. Format RDF/XML memiliki dukungan yang paling baik, hanya saja kurang intuitif jika dibandingkan dengan Turtle, Functional maupun Manchester, sedangkan Turtle Functional dan Manchester tidak memiliki dukungan \emph{tool} yang baik.
\begin{itemize}
	\item contoh sintak RDF/XML \\
	\begin{lstlisting}
		<SubClassOf>
			<Class IRI="Woman"/>
			<Class IRI="Person"/>
		</SubClassOf>
	\end{lstlisting}

	\item contoh sintak functional \\
	\begin{lstlisting}
		SubClassOf( :Woman :Person )
	\end{lstlisting}
	
	\item contoh sintak manchester
	\begin{lstlisting}
		Class: Woman
			SubClassOf: Person
	\end{lstlisting}

	\item contoh sintak turtle
		\begin{lstlisting}
			:Woman rdfs:subClassOf :Person .
		\end{lstlisting}
\end{itemize}

\subsection{Profil OWL 2}
Seperti yang telah dijelaskan sebelumnya bahwa OWL 1 memiliki tiga sub-bahasa, namun dalam praktiknya ketiga sub-bahasa tersebut ternyata belum cukup untuk memenuhi kebutuhan yang ada. \citet{patel} mengungkapkan beberapa permasalahan dalam \emph{real world application} yang diadapi para pengembang.

Berdasarkan fakta tersebut, maka OWL 2 membagi sub-bahasa kedalam tiga bagian yaitu:
\begin{itemize}
	\item OWL 2 EL
	\item OWL 2 QL
	\item OWL 2 RL
\end{itemize}

Masing-masing profil dibatasi oleh batasan sintaks \emph{(syntactic restriction)}. Perlu diketahui bahwa sub-bahasa OWL 2 ini berdasarkan pada OWL-DL, sehingga semua ekspresi yang diyatakan valid pada OWL 2 EL misalnya, secara otomatis akan valid juga untuk OWL-DL. Gambar \ref{fig:owl_2_profile} menunjukkan diagram venn relasi antar sub-bahasa pada OWL 2.

\begin{figure}[h]
	\centering
	\includegraphics[scale=0.4]{owl_2_profile}
	\caption{Diagram venn profil OWL 2}
	\label{fig:owl_2_profile}
\end{figure}
\section{Ontologi Reasoning}
Proses \emph{reasoning} adalah proses untuk mendapatkan \emph{statement} yang terdapat dalam ontologi namun tidak dinyatakan secara implisit. \citet*{antoniou} menyebutkan beberapa hal yang dapat dihasilkan melalui proses reasoning adalah:
\begin{itemize}
	\item Keanggotaan kelas \emph{(class membership)}. Menentukan apakah sebuah \emph{instance} merupakan anggota dari sebuah kelas. Penentuan keanggotaan ini dilakukan dengan cara memeriksa properti yang dimiliki oleh \emph{instance} tersebut.
	\item Klasifikasi . Apabila terdapat kelas bebek yang merupakan sub-kelas motor dan kelas motor sub-kelas dari kendaraan, maka dapat diperoleh \emph{statement} bahwa kelas bebek adalah sub-kelas dari kendaraan.
	\item Konsistensi dari sebuah ontologi. Untuk menentukan apakah sebuah ontologi konsisten seacara logika dapat pula dilakukan dengan menggunakan proses rasoning. Sebagai contoh, misalnya terdapat dua buah kelas mahasiswa dan dosen yang dinyatakan \emph{disjoint} dan terdapat satu buah \emph{instance} syamsul yang merupakan anggota dari kelas dosen dan mahasiswa, maka ontologi tersebut dikatakan tidak konsisten.
	\item Kesetaraan kelas \emph{equivalence of classes}. Reasoning juga dapat digunakan untuk menentukan ekivalensi kelas, misalnya terdapat kelas kajur yang dinyatakan ekivalen dengan kelas karyawan dan kelas karyawan ekivalen dengan dosen, maka kelas kajur dengan dosen juga ekivalen. 
\end{itemize}
\input{./parts/BAB_3/ontology_merging}
\section{\textit{Question Answering}}
Semakin banyaknya sumber informasi yang tersedia di internet menyebabkan semakin sulitnya mendapatkan infromasi yang relevan, informasi yang diberikan oleh mesin pencari konvensional saat ini hanya berupa tautan ke laman yang mengandung meta-keyword yang mirip dengan kata kunci yang dimasukkan, pengguna harus membaca terlebih dahulu laman web yang diberikan oleh mesin pencari yang tentu saja akan membutuhkan waktu. Hal ini menjadi tidak efisien terutama apabila yang akan dicari adalah informasi-informasi sederhana seperti informasi cuaca, alamat sebuah instansi dan lain sebagainya. 

Tujuan dari \emph{Question Answering (QA)} adalah untuk mencari jawaban atas pertanyaan pengguna dalam bentuk terstruktur maupun tidak terstruktur \citep*{moussa_kader}. Pengguna memasukkan pertanyaan dalam bentuk bahasa alami, sistem kemudian akan memproses pertanyaan dan akan menyajikan jawabannya dalam bentuk jawaban singkat hasil pemrosesan.

\citet*{ramprasath_hariharan} menyebutkan arsitektur sebuah sistem QA secara umum terdiri dari beberapa modul yiatu :
\begin{enumerate}
	\item Antarmuka pertanyaan \emph{(Query Interface)} \\
		Modul ini berfungsi sebagai penjembatan antara pengguna dengan sistem. Di sinilah pengguna memasukkan kata kunci pencariannya dalam bentuk bahasa alami.
	\item Modul Analisa Pertanyaan \emph{(Query Analyzer)} \\
		Modul ini akan menganalisa subjek, predikat dan objek dari kata kunci yang dimasukkan oleh pengguna.
	\item Modul Klasifikasi Pertanyaan \emph{(Question Classification)} \\
		Bagian ini berfungsi untuk memeriksa tipe dari pertanyaan seperti misalnya \emph{siapa} secara eksplisit menanyakan tentang orang, \emph{kapan}, berhubungan dengan waktu dan lain sebagainya. Klasifikasi ini akan digunakan sebagai pegangan pada tahapan validasi jawaban, apakah jawaban yang diberikan sesui dengan pertanyaan.
	\item Pembentukan Query \emph{(Query Reformulation)} \\
		Modul ini memegang peranan cukup penting, karena bagian ini yang bertanggungjawab atas keabsahan jawaban yang dihasilkan.
	\item Modul Pencari \emph{(Search Engine)} \\ 
		Modul ini digunakan untuk mencari jawaban dari pertayaan yang dihasilkan oleh modul query reformulation. Jawaban akan dicari pada sumber pengetahuan yang telah ditetukan, misalnya laman web ataupun modul basis pengetahuan tertentu seperti ontologi dan lain sebagainya.
	\item Module Ekstraksi \emph{(Answering Extractor)} \\ 
		Kandidat jawaban yang dihasilakan oleh modul search engine kemudian akan dikirimkan ke bagian ini, dimana kandidat jawaban yang umumnya berupa dokumen akan diekstraksi dan selanjutkan akan dikirimkan ke modul penyaring.
	\item Modul Penyaring Jawaban \emph{(Answer Filtering)} \\
		Modul ini akan menyaring kandidat jawaban hasil ekstraksi yang relevan dengan pertanyaan yang diberikan.
	\item Validasi Jawaban \emph{(Answer Validation)} \\ 
		Sebelum jawaban ditampilkan kepada pengguna, terlebih dahulu jawaban akan divalidasi berdasarkan klasifikasi pertanyaan yang telah ditentukan pada modul klasifikasi pertanyaan.
	\item Ontology Merging \\
		Mesikipun multi-ontologi memiliki kelebihan pada kayanya konsep pengetahuan yang didapat namun demikian oleh karena sangat dimungkinkan masing-masing ontologi yang menjadi sumber informasi ini memiliki struktur yang berbeda, sehingga arsitektur multi-ontologi seperti ini memunculkan tantangan baru yaitu bagaimana menggali informasi dari berbagi ontologi yang tersebar tersebut. Salah satu solusinya adalah dengan menggunakan metode merging.
\end{enumerate}
\citet*{choi} mendefinisikan ontology-merging sebagai proses pembentukan sebuah ontologi yang mendefinisikan sebuah makna tertentu dari dua buah atau lebih ontologi yang mendefinisikan makna tersebut dengan cara dan bahasa yang berbeda, sehingga diharapkan akan terbentuk sebuah ontologi yang memiliki keseragaman bahasa untuk sebuah konsep tertentu. Ontologi hasil merging memiliki menyimpan informasi dari masing-masing ontologi sumber, namun memiliki keunikan dan bukan merupakan pengganti dari ontologi sumber infromasinya.