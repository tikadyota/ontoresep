\section{Skenario Pengujian}
Pengujian terhadap sistem \emph{question answering} yang dikembangkan dalam penelitian ini meliputi pengujian sistem secara umum dan pengujian ontologi. Pengujian secara umum dilakukan dengan tujuan untuk mengetahui kemampuan sistem dalam menjawab pertanyaan-pertanyaan yang telah dikumpulkan dari responden, sedangkan pengujian ontologi dibagi menjadi dua yaitu pengujian masing-masing ontologi dan pengujian ontologi yang telah mengalami proses \emph{merging}

Pengujian ontologi secara individu bertujuan untuk memastikan konsistensi ontologi yang bersangkutan, sedangkan pengujian pada tahahapan setelah mengalami proses merging bertujuan untuk memastikan pula ontologi yang telah di-\emph{merge} tidak mengalamai inkonsistensi akibat proses \emph{merging}. Pengujian konsistensi ontologi penting untuk memastikan akurasi jawaban yang dihasilkan 