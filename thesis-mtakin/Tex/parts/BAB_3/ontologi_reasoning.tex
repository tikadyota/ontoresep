\section{Ontologi Reasoning}
Proses \emph{reasoning} adalah proses untuk mendapatkan \emph{statement} yang terdapat dalam ontologi namun tidak dinyatakan secara implisit. \citet*{antoniou} menyebutkan beberapa hal yang dapat dihasilkan melalui proses reasoning adalah:
\begin{itemize}
	\item Keanggotaan kelas \emph{(class membership)}. Menentukan apakah sebuah \emph{instance} merupakan anggota dari sebuah kelas. Penentuan keanggotaan ini dilakukan dengan cara memeriksa properti yang dimiliki oleh \emph{instance} tersebut.
	\item Klasifikasi . Apabila terdapat kelas bebek yang merupakan sub-kelas motor dan kelas motor sub-kelas dari kendaraan, maka dapat diperoleh \emph{statement} bahwa kelas bebek adalah sub-kelas dari kendaraan.
	\item Konsistensi dari sebuah ontologi. Untuk menentukan apakah sebuah ontologi konsisten seacara logika dapat pula dilakukan dengan menggunakan proses rasoning. Sebagai contoh, misalnya terdapat dua buah kelas mahasiswa dan dosen yang dinyatakan \emph{disjoint} dan terdapat satu buah \emph{instance} syamsul yang merupakan anggota dari kelas dosen dan mahasiswa, maka ontologi tersebut dikatakan tidak konsisten.
	\item Kesetaraan kelas \emph{equivalence of classes}. Reasoning juga dapat digunakan untuk menentukan ekivalensi kelas, misalnya terdapat kelas kajur yang dinyatakan ekivalen dengan kelas karyawan dan kelas karyawan ekivalen dengan dosen, maka kelas kajur dengan dosen juga ekivalen. 
\end{itemize}