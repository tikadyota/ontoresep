\section{Sistematika Penulisan}
Penulisan laporan hasil penelitian ini terbagi dalam tujuh bab dengan rincian sebagai berikut:
\begin{enumerate}
	\item Bab I Pendahuluan\\
	Bab I menjelaskan tentang latar belakang, rumusan masalah, tujuan penelitian, manfaat penelitian metodologi penelitian serta sistematika penulisan.
	\item Bab II Tinjauan Pustaka\\
	Bab II menjelaskan tentang penelitian-penelitian sebelumnya yang berkaitan dengan bidang penelitian ini yaitu sistem \emph{question answering} dengan menggunakan teknologi semantik web dan ontologi sebagai sumber pengetahuannya. Selain itu, pada bab ini juga membahas mengenai perbedaan penelitian terdahulu dengan penelitian ini.
	\item Bab III Landasan Teori\\
	Bab III membahas mengenai teori-teori yang menunjang penelitian ini seperti teori pemrosesan bahasa alami, semantik web khususnya teori tentang OWL ontologi dan reasoning serta teori tentang sistem \emph{question answering}.
	\item Bab IV Analisis dan Rancangan Sistem\\
	Bab ini membahas mengenai rancangan sistem yang dikembangkan dalam penelitian ini, meliputi arsitektur sistem, perancangan ontologi, perancangan proses reasoning hingga perancangan antar muka sistem.
	\item Bab V Implementasi Sistem\\
	Bab V membahas tentang implementasi dalam bentuk program atas rancangan yang telah dibuat pada bab sebelumnya. Implementasi di sini berupa implementasi pembuatan ontologi, pemrosesan bahasa, parser ontologi hingga implementasi ontologi reasoning dengan menggunakan OWL API dan Pellet sebagai \emph{Application Programming Interface (API)}.
	\item Bab VI Pengujian Sistem\\
	Bab ini membahas mengenai skenario pengujian dan pengujian sistem yang telah dibangun. Bab ini juga membahas mengenai hasil pengujian yang dilakukan terhadap sistem untuk kemudian nantinya dijadikan bahan pada bab selanjutnya.
	\item Bab VII Kesimpulan dan Saran\\
	Bab VII merangkum hasil pengujian yang telah dilakukan serta memberikan catatan-catatan mengenai apa saja yang perlu dikembangkan lebih lanjut pada penelitian ini untuk penelitian selanjutnya.
\end{enumerate}