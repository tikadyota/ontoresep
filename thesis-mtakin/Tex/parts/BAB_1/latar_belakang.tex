\section{Latar Belakang}
Provinsi Nusa Tenggara Barat merupakan salah satu provinsi di Indonesia bagain timur yang berupa kepulauan dimana terdapat dua buah pulau terpisah yaitu pulau Lombok dan pulau Sumbawa dengan total luas daerah 20.153,15 $km^{2}$. Disamping kedua pulau tersebut, terdapat pula beberapa pulau kecil di sekitarnya. Nusa Tenggara Barat memiliki delapan buah kabupaten serta dua buah kota madya. Masing-masing kabupaten dan kota telah dilengkapi dengan website sehingga masyarakat dapat dengan mudah mengakses informasi mengenai kabupaten/kota yang bersangkutan. Informasi-informasi yang disajikan dalam website adalah berupa informasi pokok yang berkaitan dengan daerah seperti informasi geografis, kependudukan serta potensi dan sumber daya daerah.

Nusa Tenggara Barat memiliki berbagi macam sektor andalan untuk dijadikan sebagai sumber pendapatan asli daerah (PAD) seperti pertanian, peternakan dan pariwisata. Sektor pariwisata merupakan sektor yang tengah mengalami perkembangan yang cukup pesat. Pemerintah daerah terus berupaya untuk meningkatkan jumlah kunjungan wisatawan lokal maupun internasional, salah satu upaya yang dikalukan adalah dengan meluncurkan program Visit Lombok Sumbawa pada tahun 2012. Saat ini masih banyak potensi wisata yang belum dikelola dengan baik, hal ini terlihat dari masih minimnya sarana dan prasarana pendukung seperti akses jalan, transportasi serta hotel atau penginapan yang tersedia. Selain sektor pariwisata, sektor pertanian juga memiliki potensi yang sangat besar. Tembakau merupakan komoditas andalan sektor pertanian. Berdasarkan data yang disampaikan oleh \citep*{nur_apriana}, pada tahun 2009 produksi tembakau virginia di Nusa Tenggara Barat mencapai 42.922 ton yang kemudian pada tahun 2011 meningkat sekitar 48.000 ton.

Fakta tersebut menunjukkan bahwa Nusa Tenggara Barat merupakan daerah yang sangat potensial bagi para wisatawan maupun investor yang ingin berkunjung dan berinvestasi baik di sektor pariwisata maupun sektor-sektor lainnya, untuk itu sangat penting bagi masing-masing kabupaten untuk menyediakan informasi yang terkait misalnya informasi mengenai moda transportasi untuk daerah wisata terntentu, obyek wisata yang ditawarkan, peluang investasi serta informasi mengenai dinas-dinas terkait jika ingin melakukan investasi. Informasi tersebut dapat disajikan secara online melalu website sehingga dapat diakses dimanapun dan kapanpun.

Setiap instansi di tingkat Kabupaten/kota maupun tingkat provinsi telah dilengkapi dengan website masing-masing yang bertujuan untuk menyajikan informasi data daerah yang dikelola oleh instansi tersebut. Masing-masing website dikelola secara mandiri oleh instansi yang bersangkutan. Tata kelola informasi seacara mandiri seperti ini menimbulkan masalah ketika seseorang ingin mencari informasi yang bersumber dari berbagai macam instansi, seperti misalnya ``pariwisata lombok timur'' dimana pengguna mengarapkan informasi yang komprehensif mengenai pariwisata Lombok Timur, misalnya lokasi daerah wisata, jenis wisata yang ditawarkan hingga peluang investasi di bidang pariwisata. Untuk mengatasi permasalahan tersebut, salah satu solusi yang dapat digunakan adalah dengan menggunakan sistem \textit{Question Answering} \citep{zadeh}.

Sistem \textit{Question Answering} (QA) sendiri dihadapkan pada beberapa tantangan yaitu pengetahuan \textit{(Knowledge)}, konsep relevansi \textit{(Concept of Relevance)} serta pengambilan kesimpulan berdasarkan persepsi dari sebuah informasi \textit{(Perception-based Information)} \citep{zadeh}.

Semantik web menjanjikan berbagai macam kelebihan terhadap sistem \textit{Question Answering}, terutama pada pendefinisian konsep dan \textit{knowledge} terhadap sebuah domain pengetahuan tertentu sehingga kemampuan QA dalam mendefinisikan persepsi dan kesimpulan menjadi lebih baik \citep*{guo_zhang}. Pendefinisian sebuah konsep atau domain dalam semantik web menggunakan ontologi

Berbagai model QA berbasis semantik web dengan menggunakan ontologi sebagai basis pengetahuannya telah banyak dikembangkan seperti \citet*{guo_zhang} dengan menggunakan Agent-Based dan \citet*{angele} dalam bidang kimia. Sedangkan sistem QA berbasis multi-ontologi sebelumnya pernah dikembangkan pula oleh \citet*{lopez}.

Berdasarkan fakta mengenai potensi yang dimiliki propinsi Nusa Tenggara Barat yang telah disebutkan diatas, dimungkinkan untuk membentuk beberapa domain pengetahuan yang dapat digunakan sebagai basis pengetahuan untuk membangun sistem question answering, sehingga diharapkan sistem mampu memberikan informasi secara lebih spesifik namun dengan cakupan informasi yang lebih luas.
% sebutkan kenapa multi ontologi
% sebutkan tantangan merging

% Web Ontologi Language versi 2 (OWL 2.0) menawarkan kekuatan pendefinisian pengetahuan......... sehingga kita dapat mendefinisikan sebuah konsep pengetahuan dengan lebih kompleks dan kaya.
% berikan penjelasan mengenai kelebihan menggunakan OWL 2, sehingga tujuan penelitian nanti melibatkan kelebihan OWL 2
